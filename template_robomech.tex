\documentclass{jarticle}
\usepackage{robomech}
\usepackage{graphicx}

%追加
% \usepackage{flushend}  % 最終ページの2カラムの左右の高さを揃える
% \usepackage[dvipdfmx]{graphicx}
\usepackage{subcaption}
\captionsetup[figure]{justification=centering}
\captionsetup[table]{justification=centering}
\usepackage[ipa]{pxchfon}
% \usepackage[dvipdfmx,setpagesize=false]{hyperref}



\begin{document}
\makeatletter
\title{視覚と行動のend-to-end学習により\\経路追従行動をオンラインで模倣する手法の提案}
{― データセット収集密度の動的調整による学習の効率化 ―}
{A proposal for an online imitation method of
path-tracking\\ behavior by end-to-end learning of vision and action}
{- Efficiency improvement of learning by dynamic adjustment of dataset collection density -}

\author{
\begin{tabular}{c}
 \hspace*{0.3zw}○学\hspace{1zw}今井悠月 (千葉工大)\hspace{2zw}正\hspace{1zw}上田隆一 (千葉工大)\hspace{2zw}正\hspace{1zw}林原靖男(千葉工大)\\
 \end{tabular}
 \vspace{1zh} \\
 \begin{tabular}{l}
 {\hspace*{4zw}\small Yuzuki IMAI, Chiba Institute of Technology, s20c1015as@s.chibakoudai.jp}\\
 {\hspace*{4zw}\small Ryuichi UEDA, Chiba Institute of Technology}\\
 {\hspace*{4zw}\small Yasuo HAYASHIBARA, Chiba Institute of Technology}\\
\end{tabular}
}
\makeatother

\abstract{ \small 
We have proposed an online imitation method of path-tracking behavior based on end-to-end learning of vision and action.
In recent years, many studies of autonomous movement using end-to-end learning have been reported. However, these studies have also observed deviations from the target path.
One of the possible reasons for this is the lack of training data for returning to the path.
In this paper, we perform end-to-end learning to follow a route generated by a map-based navigation system. The dataset was collected in two ways, one is to learn only the area around the route and the other is to learn the state away from the route, and the generated path-tracking behaviors were analyzed.
In addition, we proposed a new method of collecting teacher data to reinforce the behavior of returning to the path, and verified the effectiveness of the method by experiments using a simulator.
}

\date{} % 日付を出力しない
\keywords{Autonomous mobile robot,  Navigation,  End-to-end learning,  Dataset}

\maketitle
\thispagestyle{empty}
\pagestyle{empty}

\small
\section{緒言\protect\\}
本研究グループは,移動ロボットにおけるナビゲーション手段の複数化を目標としている.その目標を達成するため,
従来よりカメラ画像とロボットの角速度を end-to-end 学習することで,経路追従する手法を提案
し,その有効性を検証してきた\cite{okada}\cite{okada2}\cite{kiyooka}.
本手法(以後,従来手法と呼ぶ)は,複数のセンサと地図を入力として生成した行動を,画像を入力とする
行動に模倣する.これによりロボットは,複数のセンサと地図を使用した時と同じ行動を,画像のみで行えるように
なる.よって,地図に基づく経路追従と画像に基づく経路追従の 2 つのナビゲーション手段を獲得できる.
これらを状況に応じて高い信頼性が見込まれる方を選択することで経路追従を継続できる可能性が高まる.\\
\hspace*{1zw}本研究と同様に,カメラ画像を入力として,end-to-end 学習により経路追従する手法は,いくつか提案されている.
例えば,Muller らは,人のコントローラ操作と画像を end-to-end 学習することで,オフロード環境で
障害物を回避して走行できることを確認した\cite{off_load}.また,Moridian らは,人のコントローラ操作と
画像,測域センサのデータを end-to-end 学習することで,一定の経路を追従できることを
確認した\cite{Moridian}.さらに,Bojarski らは,画像と人が操作したステアリングの角度を end-to-end 学習する
ことで,自動車の自律移動手法を提案した\cite{Bojarski}.ただし,これらの研究はすべて人の
コントローラ操作を教師データとしている.それに対して,従来手法ではルールベースの制御器の出力である角速度を
教師データとしている.そのため,データセットの収集に人の操作が不要であることや,ロボットが目標経路に復帰
する行動を収集できるため,経路追従の継続に有効である\cite{imai}などの利点がある.ここで,従来手法は
オンラインで学習する.したがって,データセットの収集と学習をするには,ロボットを目標経路に沿って
走行させ続ける必要がある.これは作業の負担が大きく問題となっている.


\section{従来手法\protect\\}
従来手法の\\

\subsection{地図ベースの経路追従行動の模倣学習}
地図ベースの\\

\subsection{訓練済みモデルを用いた経路追従}



\footnotesize
\begin{thebibliography}{99}

\bibitem{okada}
岡田眞也, 清岡優祐, 上田隆一, 林原靖男,“視覚と行動の end-
to-end 学習により経路追従行動をオンラインで模倣する手法
の提案”,計測自動制御学会 \textit{SI} 部門講演会 \textit{SICE-SI2020} 予稿
集, pp.1148-1152, 2020.

\bibitem{okada2}
岡田眞也, 清岡優祐, 春山健太, 上田隆一, 林原靖男, “視覚と
行動の end-to-end 学習により経路追従行動 をオンラインで
模倣する手法の提案 -経路追従行動の修正のためにデータセッ
トを動的に追加する手法の検討-”, 計測自動制御学会 SI 部門
講演会 SICE-SI2021 予稿集, pp.1066-1080, 2021.

\bibitem{kiyooka}
清岡優祐, 岡田眞也, 岩井一輝, 上田隆一, 林原靖男, “視覚と行動の end-
to-end 学習により経路追従行動 をオンラインで模倣する手法の提案 -データセットと
生成された経路追従行動の解析-”, 計測自動 制御学会 SI 部門講演会 SICE-SI2021 
予稿集, pp.1071-1075, 2021.

\bibitem{off_load}
U. Muller, J. Ben, E. Cosatto, B. Flepp, and Y. Cun.
“Off-Road Obstacle Avoidance through End-to-End Learning.”
Advances in neural information processing systems,
Vol. 18, 2005.

\bibitem{Moridian}
Moridian, Barzin, Anurag Kamal, and Nina Mahmoudian. 
“Learning Navigation Tasks from Demonstration for Semi-autonomous Remote
Operation of Mobile Robots.” 2018 IEEE International Symposium on
Safety, Security, and Rescue Robotics (SSRR), pp.1-8, 2018.

\bibitem{Bojarski}
Bojarski, Mariusz, \textit{et al}.,
“End to end learning for self-driving cars,” arXiv:1604.08316, 2016.

\bibitem{imai}
今井悠月 , 清岡優祐 , 春山健太 , 上田隆一 , 林原靖男 . ” 視覚と行動の end-to-end 学習によ
り経路追従行動をオンラインで模倣する手法の提案
ー経路への復帰行動の解析と復帰
行動を強化する教師データ収集法の検討ー . 日本機械学会ロボティクス・メカトロニクス
講演会 ’23 予稿集 , 2P2-G05(2023).

\end{thebibliography}

\normalsize
\end{document}
