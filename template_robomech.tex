\documentclass{jarticle}
\usepackage{robomech}
\usepackage{graphicx}

%追加
% \usepackage{flushend}  % 最終ページの2カラムの左右の高さを揃える
% \usepackage[dvipdfmx]{graphicx}
\usepackage{subcaption}
\captionsetup[figure]{justification=centering}
\captionsetup[table]{justification=centering}
\usepackage[ipa]{pxchfon}
\usepackage[dvipdfmx,setpagesize=false]{hyperref}



\begin{document}
\makeatletter
\title{視覚と行動のend-to-end学習により\\経路追従行動をオンラインで模倣する手法の提案}
{― データセット収集密度の動的調整による学習の効率化 ―}
{A proposal for an online imitation method of
path-tracking\\ behavior by end-to-end learning of vision and action}
{- Efficiency improvement of learning by dynamic adjustment of dataset collection density -}

\author{
\begin{tabular}{c}
 \hspace*{0.3zw}○学\hspace{1zw}今井悠月 (千葉工大)\hspace{2zw}正\hspace{1zw}上田隆一 (千葉工大)\hspace{2zw}正\hspace{1zw}林原靖男(千葉工大)\\
 \end{tabular}
 \vspace{1zh} \\
 \begin{tabular}{l}
 {\hspace*{4zw}\small Yuzuki IMAI, Chiba Institute of Technology, s20c1015as@s.chibakoudai.jp}\\
 {\hspace*{4zw}\small Ryuichi UEDA, Chiba Institute of Technology}\\
 {\hspace*{4zw}\small Yasuo HAYASHIBARA, Chiba Institute of Technology}\\
\end{tabular}
}
\makeatother

\abstract{ \small 
We have proposed an online imitation method of path-tracking behavior based on end-to-end learning of vision and action.
In recent years, many studies of autonomous movement using end-to-end learning have been reported. However, these studies have also observed deviations from the target path.
One of the possible reasons for this is the lack of training data for returning to the path.
In this paper, we perform end-to-end learning to follow a route generated by a map-based navigation system. The dataset was collected in two ways, one is to learn only the area around the route and the other is to learn the state away from the route, and the generated path-tracking behaviors were analyzed.
In addition, we proposed a new method of collecting teacher data to reinforce the behavior of returning to the path, and verified the effectiveness of the method by experiments using a simulator.
}

\date{} % 日付を出力しない
\keywords{Autonomous mobile robot,  Navigation,  End-to-end learning,  Dataset}

\maketitle
\thispagestyle{empty}
\pagestyle{empty}

\small
\section{緒言\protect\\}
近年\\


\section{従来手法\protect\\}
従来手法の\\

\subsection{地図ベースの経路追従行動の模倣学習}
地図ベースの\\

\subsection{訓練済みモデルを用いた経路追従}



\footnotesize
\begin{thebibliography}{99}

\bibitem{Shinjuku98}
新宿大五朗,渋谷次郎,東京 学,``キャスティングマニピュレーションに関する研究(第1報,可変長の紐状柔軟リンクを有するマニピュレータの提案とそのスイング制御法)'',{\it 機論C編}, vol.64-626, pp.3854--3861, 1998.

\bibitem{Shinjuku99}
Shinjuku, D., Shibuya, J. and Tokyo, M., ``Swing Motion Control of Casting Manipulation,'' {\it IEEE Control Systems}, vol.19-4, pp.56--64, 1999.

\end{thebibliography}

\normalsize
\end{document}
